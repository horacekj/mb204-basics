\documentclass[12pt,a4paper]{article}
\usepackage[unicode,colorlinks=true]{hyperref}
\usepackage[czech]{babel}
\usepackage[utf8]{inputenc}
\usepackage[T1]{fontenc}
\usepackage{lmodern}
\usepackage{graphicx}
\usepackage{amssymb}
\usepackage{amsmath}
\usepackage{amsfonts}
\usepackage{enumitem}
\textwidth 16cm \textheight 24.6cm
\topmargin -1.3cm
\oddsidemargin 0cm

\pagestyle{plain}
\setcounter{page}{1}
\pagenumbering{arabic}

\begin{document}
\title{\vspace{-1.5cm}Základy MB204}
\author{Jan Horáček; me@apophis.cz}
\maketitle

\begin{enumerate}[leftmargin=*]
	\item \textbf{Počet prvočísel}\\
		$\pi(x)$ udává počet prvočísel menších nebo rovných číslu $x \in \mathbb{R}$. \\
		\[ \pi(x) \sim \frac{x}{\ln{x}} \]

%	\item Möbiova funkce.\\
%		Rozložme přirozené číslo $n$ na prvočísla: $n = p_1^{\alpha_1} \cdots p_k^{\alpha_k}$.
%		Hodnotu Möbiovy funkce $\mu(n)$ definujeme rovnu $0$, pokud pro nekteré
%		$i$ platí $\alpha_i > 1$ a rovnu $(-1)^k$ v~opačném případě. Dále definujeme
%		$\mu(1) = 1$.

	\item \textbf{Eulerova funkce}
		\[ \varphi(n) = |\{ a \in \mathbb{N} \mid 0 < a \le n, (a,n) = 1\}| \]
		Eulerova funkce v~bodě $n$ je počet přirozených čísel v~intervalu
		$\langle 1..n \rangle$, která jsou s~číslem~$n$ nesoudělná.

		$\varphi(1) = 1$, $\varphi(5) = 4$, $\varphi(6) = 2$, $\varphi(p) = p-1$.

		\[ \varphi(n) = n \cdot \left( 1 - \frac{1}{p_1} \right) \cdot
			\left( 1 - \frac{1}{p_2} \right) \cdots \left( 1 - \frac{1}{p_k} \right) \]

		Funkce $\varphi$ je multiplikativní pro nesoudělná $a$, $b$:
		\[ \varphi(a \cdot b) = \varphi(a) \cdot \varphi(b). \]
		\[ \varphi(p^\alpha) = p^\alpha - p^{\alpha-1} = (p - 1) \cdot p^{\alpha-1}\]
		\[ \varphi(p_1^{\alpha_1} \cdots p_k^{\alpha_k}) = (p_1 - 1) \cdot
			p_1^{\alpha_1-1} \cdots (p_k - 1) \cdot p_k^{\alpha_k-1} \]
		\[ \varphi(n) = n \cdot \left( 1 - \frac{1}{q_1} \right) \cdots
			\left( 1 - \frac{1}{q_k} \right) \]

	\item \textbf{Malá Fermatova věta}\\
		Nechť $a \in \mathbb{Z}$, $p$ je prvočíslo a $p \nmid a$. Pak
		\[ a^{p-1} \equiv 1 \pmod{p} \]
		Věta poskytuje částečnou podmínku prvočíselnosti.

	\item \textbf{Eulerova věta}\\
		Nechť $a \in \mathbb{Z}$, $m \in \mathbb{N}$ a $(a,m) = 1$. Pak
		\[ a^{\varphi(m)} \equiv 1 \pmod{m} \]
		Je zobecněním Fermatovy věty.

	\item \textbf{Úplná soustava zbytků}\\
		Úplná soustava zbytků modulo $m$ je libovolná $m$-tice čísel po dvou
		nekongruentních modulo $m$. Nejčastěji $0, 1, \dots, m-1$, nebo
		(pro lichá $m$) $-\frac{m-1}{2}, \dots, -1, 0, 1, \dots, \frac{m-1}{2}$.

	\item \textbf{Redukovaná soustava zbytků}\\
		Redukovaná soustava zbytků modulo $m$ je libovolná $\varphi(m)$-tice
		čísel nesoudělných s~$m$ a po dvou nekongruentních modulo $m$.

	\item \textbf{Řád čísla modulo $m$}\\
		Nechť $a \in \mathbb{Z}$, $m \in \mathbb{N}$ a $(a,m) = 1$. Řádem čísla
		\textit{a modulo m} rozumíme nejmenší přirozené číslo $n$ splňující:
		\[ a^n \equiv 1 \pmod{m} \]

	\item \textbf{Primitivní kořen}\\
		Nechť $m \in \mathbb{N}$. Celé číslo $g \in \mathbb{Z}$, $(g,m) = 1$
		nezveme \textit{primitivním kořenem modulo m}, pokud je jeho řád modulo
		$m$ roven $\varphi(m)$.

		Pro lichá prvočísla existují primitivní kořeny.

		Umocňováním primitivního kořene modulo $p$ dostaneme všechny prvky
		redukované soustavy zbytků modulo $k$.

	\item \textbf{Wilsonova věta}\\
		Udává nutnou i postačující podmínku prvočíselnosti.

		Přirozené číslo $n > 1$ je prvočíslo, právě když
		\[ (n-1)! \equiv -1 \pmod{n}. \]
		Bohužel dosud není známo, jak rychle počítat modulární faktoriál velkých
		čísel.

	\item \textbf{Čínská zbytková věta}\\
		Nechť $m_1, \ldots, m_k \in \mathbb{N}$ jsou po dvou nesoudělná a
		$a_1, \ldots, a_k \in \mathbb{Z}$.

		Pak soustava
		\begin{align*}
			x &\equiv a_1 \pmod{m_1}\\
			&\vdots\\
			x &\equiv a_k \pmod{m_k}
		\end{align*}

		má jediné řešení modulo $m_1 \cdot m_2 \cdots m_k$.

	\item \textbf{Binomické kongruence}
		\[ x^n -a \equiv 0 \pmod{m} \]

	\begin{enumerate}
		\item \textbf{Mocninné zbytky}\\
			Nechť $m \in \mathbb{N}$, $a \in \mathbb{Z}$, $(a,m) = 1$. Číslo $a$
			nazveme \textit{n-tým mocninným zbytkem modulo m}, pokud je kongruence
			\[ x^n \equiv a \pmod{m} \]
			řešitelná. V opačném případě nazveme $a$ \textit{n-tým mocninným
			nezbytkem modulo m}.

			V~redukované soustavě zbytků modulo $m$ je stejný počet
			kvadratických zbytků a nezbytků.

		\item \textbf{Existence řešení binomických kongruencí}\\
			Buď $m \in \mathbb{N}$ takové, že modulo $m$ existují primitivní
			kořeny.  Dále nechť $a \in \mathbb{Z}$, $(a,m) = 1$. Pak kongruence
			$x^n \equiv a \pmod{m}$ je řešitelná (tj. $a$ je n-tý mocninný
			zbytek modulo $m$), právě když
			\[ a^{\varphi(m)/d} \equiv 1 \pmod{m} \]
			kde $d = (n, \varphi(m))$ je počet řešení.
	\end{enumerate}

	\item \textbf{Polynomiální kongruence}
	\begin{enumerate}
		\item \textbf{Henselovo lemma}\\
			udává postup pro řešení kongruencí modulo mocnina
			prvočísla.

			Nechť $p$ je prvočíslo, $f(x) \in \mathbb{Z}[x]$, $a \in
			\mathbb{Z}$ je takové, že $p \mid f(a)$, $p \nmid f'(a)$. Pak
			platí: pro každé $n \in \mathbb{N}$ má soustava
			\begin{align*}
				x &\equiv a \pmod{p}\\
				f(x) &\equiv 0 \pmod{p^n}
			\end{align*}
			právě jedno řešení modulo $p^n$. Viz příklad 4/10.

		\item \textbf{Řešitelnost kvadratických kongruencí}\\
			Podmínka řešitelnosti kongruence
			\[ ax^2 + bc + c \equiv 0 \pmod{m} \]
			je ekvivalentní podmínce řešitelnosti binomické kongruence
			\[ x^2 \equiv a \pmod{p}. \]

		\item \textbf{Legendreův symbol}\\
			Nechť $p$ je liché prvočíslo.

			\[ \genfrac(){}{0}{a}{b} = \left\{
				\begin{array}{ll}
					1 & p \nmid a, a \text{ je kvadratický zbytek modulo } p \\
					0 & p \mid a \\
					-1& p \nmid a, a \text{ je kvadratický nezbytek modulo } p
				\end{array}
			\right. \]

			Nechť $p$ je liché prvočíslo a $a, b \in \mathbb{Z}$ jsou libovolná.
			Pak platí:

			\begin{enumerate}
				\item $\genfrac(){}{1}{a}{p} \equiv  a^{\frac{p-1}{2}} \pmod{p},$
				\item $\genfrac(){}{1}{ab}{p} = \genfrac(){}{1}{a}{p} \cdot
					\genfrac(){}{1}{b}{p},$
				\item $a \equiv b \pmod{p} \Rightarrow \genfrac(){}{1}{a}{p}
					= \genfrac(){}{1}{b}{p}.$
			\end{enumerate}

		\item \textbf{Kvadratická reciprocita po Legendreův symbol}\\
			Nechť $p$, $q$ jsou lichá prvočísla. Pak:

			\begin{enumerate}
				\item $\genfrac(){}{1}{-1}{p} = (-1)^{\frac{p-1}{2}}$
				\item $\genfrac(){}{1}{2}{p} = (-1)^{\frac{p^2-1}{8}}$
				\item $\genfrac(){}{1}{q}{p} = \genfrac(){}{1}{p}{q} \cdot
					(-1)^{\frac{p-1}{2} \cdot \frac{q-1}{2}}$
			\end{enumerate}

		\item \textbf{Jacobiho symbol}\\
			Nevýhoda Legendreova symbolu tkví v tom, že je definován jen pro
			lichá prvočísla. Tento problém řešíme zavedením \textit{Jacobiho
			symbolu}.

			Nechť $a \in \mathbb{Z}$, $b \in \mathbb{N}$, $2 \nmid b$. Nechť
			$b = p_1 \cdot p_2 \cdots p_k$ je rozklad čísla $b$ na (lichá)
			prvočísla (neskupujeme stejná prvočísla do mocniny!). Symbol
			\[ \genfrac(){}{0}{a}{b} = \genfrac(){}{0}{a}{p_1} \cdot
				\genfrac(){}{0}{a}{p_2} \cdots \genfrac(){}{0}{a}{p_k} \]

			se nazývá \textit{Jacobiho symbol}.

			Pozor: pro \textit{Jacobiho symbol} neplatí implikace
			\[ (a/b) = 1 \Rightarrow x^2 \equiv a \pmod{b} \text{ je řešitelná.} \]

		\item \textbf{Kvadratická reciprocita po Jacobiho symbol}\\
			Stejná jako pro \textit{Legendreův symbol}.

	\end{enumerate}

\end{enumerate}

\end{document}
